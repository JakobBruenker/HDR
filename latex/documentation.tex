\section{Documentation for the Program}\label{sec:doc}

To run the program, you need a \texttt{config.cfg} file in the directory you
are running it from, which contains the relative or absolute path to a
\texttt{.hdrgen} file that specifies which images should be used. You can then
run \texttt{make}. (This Makefile probably only works under UNIX-like operating
systems, because it relies on X11. The source code, however, should work on
Windows as well.) This will run the program after compiling it if necessary.

The program will then generate an image in the OpenEXR format and save it with
the same name as the \texttt{.hdrgen} file, replacing the file ending with
\texttt{.exr}.  Similarly, it will write a file containing the values of the
response curve in a file with the same name as the \texttt{.hdrgen} file,
replacing the file ending with \texttt{.txt}. It lists the values for red
pixels in the first column, the ones for green pixels in the second column, and
the ones for blue pixels in the third column.

It will also open several windows to show a tonemapped version of the HDR image
as well as 3 simulated exposures: 0.01 seconds, 0.05 seconds, and 0.33 seconds.
It will also show the recovered response curve; the required light to reach a
certain pixel value for a certain color increases towards the right, and the pixel
value reached increases towards the top. The intensities are on a logarithmic
scale.

To exit the program, simply close one of the windows it opens.
